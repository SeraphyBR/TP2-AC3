%!TEX TS-program = xelatex
%!TEX encoding = UTF-8 Unicode
%
\documentclass[a4paper,11pt]{article}
\usepackage{graphicx,url}
\usepackage[T1]{fontenc}
\usepackage[brazil]{babel}
\usepackage{a4wide}
\usepackage{fontspec}
\setmainfont{Times New Roman}
\graphicspath{{./imagens/}}

\title{\vspace{-4cm}Relatório do Trabalho Prático 2\\Arquitetura de Computadores}
\author{
    Gabriel Lopes Ferreira\\
    Luiz Junio Veloso Dos Santos\\
    Matheus Luiz Oliveira Spindula\\
    Rebeca Neto\\
}

\begin{document}

\maketitle

\begin{enumerate}
    \item \textbf{Introdução:}
        O objetivo

    \item \textbf{O simulador:}

    \item \textbf{Análises:}
        \begin{enumerate}
            \item \textbf{Teste 1}
            \newline
            Neste teste, usamos uma arquitetura superescalar com \textit{2-way Pipeline}, 
            \textit{2-line buffer}, um trace com 21 instruções, com todas as instruções 
            idênticas ('ADD R1,R2,R3'), ou seja, todas leem e escrevem no mesmo local, 
            comparamos o impacto de utilizar 1 ALU vs 2 ALU\@.
        
            \begin{figure}[!ht]
            \caption{Execução com 1 ALU}
            \centering
            \includegraphics[width=1\textwidth]{teste1-a}
            \end{figure}

            No teste acima, apesar de muitas instruções serem despachadas para o Instruction Queue,
            poucas vão para o Reservation Station (Janela Distribuida), já que existe somente 1 ALU
            e sua janela comporta somente 2 instruções e somente uma é executada em cada ciclo.
            Após sair da ALU o resultado vai para o \textit{Reorder Buffer}, para que possa ser escrito no banco
            de registradores. Como neste trace todas as instruções escrevem no mesmo local, existe um
            ``acumulo'' de instruções no buffer com resultado pronto, porém aguardando uma instrução
            anterior terminar a escrita.

            \newpage
            \begin{figure}[!ht]
            \caption{Execução com 2 ALU}
            \centering
            \includegraphics[width=1\textwidth]{teste1-b}
            \end{figure}

            Na segunda parte do teste, foi adicionada uma segunda ALU, isso fez com que um maior
            numero de instruções fossem despachadas para as \textit{Reservation Stations}, gerando
            um menor acumulo de instruções no \textit{Instruction Queue}, o fato de ter 2 ALU's faz 
            com que ocorra um paralelismo de instruções, duas em cada janela. 
            Contudo esse paralelismo fez com que houvesse um maior acumulo de instruções 
            no \textit{Reorder Buffer}.
        \end{enumerate}
\end{enumerate}

\end{document}
