%!TEX TS-program = xelatex
%!TEX encoding = UTF-8 Unicode
\documentclass[conference]{IEEEtran}
\IEEEoverridecommandlockouts
\usepackage{cite}
\usepackage[brazil]{babel}
\usepackage{amsmath,amssymb,amsfonts}
\usepackage{algorithmic}
\usepackage{hyperref}
\usepackage{graphicx}
\usepackage{fontspec}
\usepackage{textcomp}
\usepackage{xcolor}
\setmainfont{Times New Roman}
\graphicspath{{./imagens/}}
\hypersetup{
    colorlinks=true,
    linkcolor=blue,
    filecolor=magenta,
    urlcolor=blue,
    pdftitle={Relatório do Trabalho Prático II},
    bookmarks=true,
}
\urlstyle{same}
\def\BibTeX{{\rm B\kern-.05em{\sc i\kern-.025em b}\kern-.08em
    T\kern-.1667em\lower.7ex\hbox{E}\kern-.125emX}}
\begin{document}

\title{
    Relatório do Trabalho Prático II\\
    {\footnotesize Arquitetura de Computadores III}
}

\author{
    \IEEEauthorblockN{Gabriel Lopes Ferreira}
    \IEEEauthorblockA{
        \textit{PUC Minas - ICEI} \\
        Belo Horizonte, Brazil \\
        fulano@sga.pucminas.br
    }
    \and
    \IEEEauthorblockN{Luiz Junio Veloso Dos Santos}
    \IEEEauthorblockA{
        \textit{PUC Minas - ICEI} \\
        Belo Horizonte, Brazil\\
        ljvsantos@sga.pucminas.br
    }
    \and
    \IEEEauthorblockN{Matheus Luiz Oliveira Spindula}
    \IEEEauthorblockA{
        \textit{PUC Minas - ICEI} \\
        Belo Horizonte, Brazil\\
        fulano@sga.pucminas.br
    }
    \and
    \IEEEauthorblockN{Rebeca Neto}
    \IEEEauthorblockA{
        \textit{PUC Minas - ICEI} \\
        Belo Horizonte, Brazil\\
        fulana@sga.pucminas.br
    }
}

\maketitle

\begin{abstract}
    Este é um relatório dos resultados obtidos ao realizar testes
    em um Simulador de Pipeline Superescalar com objetivo de estudo
    do funcionamento desse tipo de Pipeline.
\end{abstract}

\begin{IEEEkeywords}
    Pipeline, Superescalar, Tomasulo, Simulador
\end{IEEEkeywords}

\section{Introdução}
    This document is a model and instructions for \LaTeX.
    Please observe the conference page limits.

\section{O simulador}
    O objetivo do simulador é auxiliar estudantes a entender o conceito de superescalaridade
    na arquitetura de microprocessadores. O simulador é baseado no Algoritmo de Tomasulo,
    na versão alfa o autor busca com o auxilio de avaliações e contribuições, aprimorar seu projeto.
    O simulador nos permite testar diferentes tipos de arquitetura, como aumentar o número de instruções
    no pipeline, alterar o número de unidades operacionais e o número de instruções na janela.
    Além disso podemos ver o caminho percorrido pela instrução no pipeline superescalar de uma maneira
    gráfica e também é possível ver o impacto de um programa com dependências e conflitos de
    registradores.

\section{Metodologia adotada}
    Para analisarmos a execução, realizamos diversos testes com trace padrões da ferramenta
    e trace manipulados. Também fizemos alterações na arquitetura para analisarmos o impacto
    de diferentes arquiteturas na execução de um programa, e de diferentes programas para
    visualizarmos melhor problemas como dependências de leitura pós-escrita.

\section{Análise dos testes}

\section{Análise do simulador}
    \subsection{Pontos Positivos}
        \begin{itemize}
            \item Auxilia muito na visualização de conceitos do\\
                ``superescalar'', e aprendizagem de estudantes de\\
                arquitetura de computadores.
        \end{itemize}
    \subsection{Pontos Negativos}
        \begin{itemize}
            \item A fonte usada no simulador é um pouco pequena, os elementos dele é pequeno.
            \item Falta de um possibilidade dar zoom.
            \item Não mostra um contador de ciclos.
            \item Sempre que finalizar um teste é necessário reiniciar o programa para fazer
                outro teste e novamente colocar toda a arquitetura que sera usada.
        \end{itemize}
    \subsection{Melhorias futuras}
        \begin{itemize}
            \item Implementar uma funcionalidade de zoom no simulador e aumentar a fonte
                em todo o simulador.
            \item Para melhorar o simulador seria aconselhável que uma variável que conta
                o número de ciclos seja inserida, notamos que ja existe um contador interno,
                mas que o valor dele não esta sendo mostrado para o usuário,
                modificamos o código fonte do simulador para que essa variável fosse exibida
                na tela.
                Contudo foi possível perceber que ele não possui muitos tratamentos, onde se
                o programa parar, ele continuara contando a cada click no botão ``next'',
                independentemente do que esta sendo executado.

        \end{itemize}

\section{Conclusão}

\begin{thebibliography}{00}
    \bibitem{b1} Roberto Miranda, and Eduardo Gregório,
    Superescalar Simulator (ALPHA version) based on Tomasulo's Algorithm,
    \url{https://github.com/robertomap/SuperscalarSIM}
\end{thebibliography}

\end{document}
